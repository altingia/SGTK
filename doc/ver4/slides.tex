\documentclass{beamer}

\usepackage{xcolor,colortbl}
\usepackage{polyglossia}
\usepackage{fontspec}
\usepackage{nameref}
\usepackage{ifthen}

\usefonttheme{professionalfonts}
\usetheme{Antibes}
\useoutertheme{infolines_foot}
\setbeamercovered{transparent=20}

\setbeamertemplate{blocks}[default]
\setbeamercolor{block title}{use={frametitle},fg=frametitle.fg,bg=frametitle.bg}

\usepackage[math-style=ISO,vargreek-shape=unicode]{unicode-math}
\setdefaultlanguage[spelling=modern,babelshorthands=true]{russian}
\setotherlanguage{english}

\defaultfontfeatures{Ligatures={TeX}}
\setmainfont{CMU Serif}
\setsansfont{CMU Sans Serif}
\setmonofont{CMU Typewriter Text}
\setmathfont{Latin Modern Math}
\AtBeginDocument{\renewcommand{\setminus}{\mathbin{\backslash}}}

\makeatletter
\newcommand*{\currentname}{\@currentlabelname}
\makeatother
\def\t{\texttt}

\newcommand{\cimg}[2]{%
	\begin{center}%
		\ifthenelse{\equal{#2}{}}{%
			\includegraphics[width=0.75\linewidth]{#1}
		}{%
			\includegraphics[width=#2\linewidth]{#1}
		}%
	\end{center}%
}

\title[RNA-seq scaffolding]{Scaffold graph visualization and RNA-Seq scaffolding}
\author[Chernikova Olga]{Chernikova Olga\\
	Supervisor: Andrey Prjibelski}
\institute[CAB]{Center for Algorithmic Biotechnology}
\date{07/27/2017}

\begin{document}

\begin{frame}
	\titlepage
\end{frame}

\section{Introduction}

\begin{frame}[t]{Genome assembly}
\cimg{p1.png}{0.68}
\end{frame}

\begin{frame}[t]{Genome assembly}
\cimg{p2.png}{0.68}
\end{frame}

\begin{frame}[t]{Connection by pair DNA reads}
	\begin{itemize}
		\item Pair reads:  
		\cimg{pairRead.png}{1}
		\item Finding connections by pair reads: 
		\cimg{pairReadAlig}{1}
	\end{itemize}
\end{frame}

\begin{frame}[t]{By RNA-seq reads}
\begin{center}
\cimg{rna.png}{1.24}
\end{center}
\end{frame}

\begin{frame}[t]{By RNA-seq reads}
\begin{center}
\cimg{rnaReads.png}{1.24}
\end{center}
\end{frame}

\section{Goal and tasks}
\begin{frame}[t]{Goal and tasks}
	\begin{block}{Goal}
	Building scaffolds by RNA-seq reads 
	\end{block}
	\begin{block}{Tasks}
	\begin{itemize}
		\item Building the contig graph
		\item Building scaffolds by obtained connections
		\item Creating tool for visualizing a contig graph
		\item Comparing results with other tools for building scaffolds by RNA-seq  
	\end{itemize}
\end{block}
\end{frame}	

\section{Tasks solving}
\subsection{Visualization}

\begin{frame}[t]{Visualization}
%связь между контигами, dot формат, разбиваем граф на несколько файликов. 
\cimg{examp.jpg}{0.8}
\end{frame}

\begin{frame}[t]{Visualization}
	%Какие виды связи можно визуализировать. Референс, РНК, ДНК, скаффолды. 
	\cimg{diflib.png}{1.03}
\end{frame}


\begin{frame}[t]{Visualization}
Options for filtering the graph:
\begin{itemize}
	\item by edge weight and contig size
	\item showing only parts with difference in two libs 
	\item showing only parts with one lib but without the second one
	\item showing only parts with errors
	\item etc.
\end{itemize}
	%Варианты фильтрация +  несколько слов, для чего можно использовать 
\end{frame}

\subsection{Scaffolding}

\begin{frame}[t]{Building a contig graph}
	\begin{itemize}
		\item Alignment of pair RNA-seq reads
		\item Building a contig graph on the basis of this alignment
		\item Splitting reads into two parts
		\item Alignment of these parts
		\item Building graph by these parts of the reads
		\item Saving graph
	\end{itemize}	
	\cimg{rnaReadsCon.png}{1}
\end{frame}

\begin{frame}[t]{Simplification of the graph}
	\begin{itemize}
		\item Deleting small weight edges 
		\cimg{wrong.png}{0.8}
		\only<2->{\item Edge projection}
		\only<2->{\cimg{outinline.png}{1}}
	\end{itemize}
\end{frame}

\begin{frame}[t]{Simplification of the graph}
	\begin{itemize}
		\item Deleting cycles
		\cimg{delCycl.png}{0.4}
		\only<2->{\item Fork with big difference in weight}
		\only<2->{\cimg{bigDif.png}{0.4}}
		\only<3->{\item Connecting simple paths into scaffolds}
		\only<3->{\cimg{gline.png}{1}}
	\end{itemize}
\end{frame}

\begin{frame}[t]{Сomparison} 
	C.elegans, SRR1560107
	\begin{center}
		\begin{tabular}{|l|>{\columncolor[gray]{0.8}}c|c|c|}
			\hline
			&bio\_scaffolder&P\_RNA\_scaffolder&rascaf\\
			\hline
			NG50&36855&36075&32879\\
			\hline
			NG75&17299&17188&18395\\
			\hline
			NGA50&30383&28828&27116\\
			\hline
			NGA75&12735&12489&11667\\
			\hline
			LGA50&918&955&995\\
			\hline
			misassemblies&529&621&521\\
			\hline
		\end{tabular}
	\end{center}
\end{frame}

\section{Results}
\begin{frame}[t]{Results and plans}
	\begin{block}{Results}
		\begin{itemize}
			\item Creating the program for RNA-seq scaffolding 
			\item Creating the tool for visualization a contig graph
		\end{itemize}
	\end{block}
	\begin{block}{Plans}
		\begin{itemize}
			\item Testing and comparing with other tools given bigger diversity of data
			\item Making program faster 
			\item Implementing new idea for making scaffolds 
			\item Writing manual and interface
			\item Writing an article
			\end{itemize}-
	\end{block}
\end{frame}


\section{Thank you!}
\begin{frame}{Thank you!}
    \begin{center}
        https://github.com/olga24912/bio\_scaffolder
    \end{center}
\end{frame}
\end{document}