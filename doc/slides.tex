\documentclass{beamer}

\usepackage{polyglossia}
\usepackage{fontspec}
\usepackage{nameref}
\usepackage{ifthen}

\usefonttheme{professionalfonts}
\usetheme{Antibes}
\useoutertheme{infolines_foot}
\setbeamercovered{transparent=20}

\usepackage[math-style=ISO,vargreek-shape=unicode]{unicode-math}
\setdefaultlanguage[spelling=modern,babelshorthands=true]{russian}
\setotherlanguage{english}

\defaultfontfeatures{Ligatures={TeX}}
\setmainfont{CMU Serif}
\setsansfont{CMU Sans Serif}
\setmonofont{CMU Typewriter Text}
\setmathfont{Latin Modern Math}
\AtBeginDocument{\renewcommand{\setminus}{\mathbin{\backslash}}}

\makeatletter
\newcommand*{\currentname}{\@currentlabelname}
\makeatother
\def\t{\texttt}

\newcommand{\cimg}[2]{%
	\begin{center}%
		\ifthenelse{\equal{#2}{}}{%
			\includegraphics[width=0.75\linewidth]{#1}
		}{%
			\includegraphics[width=#2\linewidth]{#1}
		}%
	\end{center}%
}

\title[Граф связей между контигами]{Построение графа связей геномных последовательностей}
\author[Черникова Ольга]{Черникова Ольга\\
	Руководитель: Пржибельский Андрей Дмитриевич}
\institute{СПб АУ РАН}
\date{Осень 2017}

\begin{document}

\begin{frame}
	\titlepage
\end{frame}

\section{Постановка задачи}

\begin{frame}[t]{Постановка задачи}
	\begin{itemize}
		\item Хочется собрать ДНК:
		\cimg{p1_1.png}{1}
		\item Реально получается собрать только какие-то его части(контиги):
		\cimg{p1_2.png}{0.25}
		\item \textbf{Цель проекта} "--- востновить порядок следования контигов:
		\cimg{p1_3.png}{1} 
	\end{itemize}
\end{frame}

\section{Методы решения}

\section{Архитектура}

\section{Результаты}

\section{Проблемы}

\section{Использованные матерьялы}

\section{Дальнейшие пути развития}

\section{Спасибо за внимание}
\begin{frame}{Спасибо за внимание}
    \begin{center}
        Репозиторий: \\ \url{https://github.com/olga24912/bio_scaffolder}
    \end{center}
\end{frame}
\end{document}
