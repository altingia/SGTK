\documentclass{beamer}

\usepackage{xcolor,colortbl}
\usepackage{polyglossia}
\usepackage{fontspec}
\usepackage{nameref}
\usepackage{ifthen}
\usepackage{multirow}
\usepackage{hhline}

\usepackage{biblatex}

\usefonttheme{professionalfonts}
\usetheme{Antibes}
\useoutertheme{infolines_foot}
\setbeamercovered{transparent=20}

\setbeamertemplate{blocks}[default]
\setbeamercolor{block title}{use={frametitle},fg=frametitle.fg,bg=frametitle.bg}

\usepackage[math-style=ISO,vargreek-shape=unicode]{unicode-math}
\setdefaultlanguage{english}

\definecolor{LightCyan}{rgb}{0.88,1,1}

\defaultfontfeatures{Ligatures={TeX}}
\setmainfont{CMU Serif}
\setsansfont{CMU Sans Serif}
\setmonofont{CMU Typewriter Text}
\setmathfont{Latin Modern Math}
\AtBeginDocument{\renewcommand{\setminus}{\mathbin{\backslash}}}

\makeatletter
\newcommand*{\currentname}{\@currentlabelname}
\makeatother
\def\t{\texttt}

\newcommand{\cimg}[2]{%
	\begin{center}%
		\ifthenelse{\equal{#2}{}}{%
			\includegraphics[width=0.75\linewidth]{#1}
		}{%
			\includegraphics[width=#2\linewidth]{#1}
		}%
	\end{center}%
}

\AtNextBibliography{\tiny}


\title[RNA-seq scaffolding]{Scaffold graph visualization and RNA-Seq scaffolding}
\author[Kunyavskaya Olga]{Kunyavskaya Olga\\
	Supervisor: Andrey Prjibelski}
\institute[CAB]{Center for Algorithmic Biotechnology}
\date{07/27/2017}

\begin{document}

\begin{frame}
	\titlepage
\end{frame}

\section{Introduction}

\begin{frame}[t]{Genome assembly}
\cimg{pic/p1.png}{0.68}
\end{frame}

\begin{frame}[t]{Genome assembly}
\cimg{pic/p2.png}{0.68}
\end{frame}

\begin{frame}[t]{Connection using DNA paired-end reads}
	\begin{itemize}
		\item Paired reads:  
		\cimg{pic/pairRead.png}{1}
		\item Finding connections using paired reads: 
		\cimg{pic/pairReadAlig}{1}
	\end{itemize}
\end{frame}

\begin{frame}[t]{Using RNA-seq reads}
\begin{center}
\cimg{pic/rna.png}{1.24}
\end{center}
\end{frame}

\begin{frame}[t]{Using RNA-seq reads}
\begin{center}
\cimg{pic/rnaReads.png}{1.24}
\end{center}
\end{frame}

\section{Goal and tasks}
\begin{frame}[t]{Goal and tasks}
	\begin{block}{Goal}
	Build scaffolds using RNA-seq reads 
	\end{block}
	\begin{block}{Tasks}
	\begin{itemize}
		\item Build the scaffold graph
		\item Build scaffolds using obtained connections
		\item Create tool for visualizing a scaffold graph
		\item Compare results with other tools
	\end{itemize}
\end{block}
\end{frame}	
\section{Tasks solving}
\subsection{Visualization}

\begin{frame}[t]{Visualization}
%связь между контигами, dot формат, разбиваем граф на несколько файликов. 
\cimg{pic/examp.jpg}{0.8}
\end{frame}

\begin{frame}[t]{Visualization}
%Какие виды связи можно визуализировать. Референс, РНК, ДНК, скаффолды. 
\cimg{pic/diflib.png}{1.03}
\end{frame}

\begin{frame}[t]{Visualization}
% Визуализация вдоль хромосомы
\cimg{pic/alongChr.png}{1}
\end{frame}

\begin{frame}[t]{Visualization}
Drawing interesting parts:
\begin{itemize}
	\item difference between two libs 
	\item one lib is present and another is not
	\item difference with reference (possible missassembly)
	\item etc
\end{itemize}
	%Варианты фильтрация +  несколько слов, для чего можно использовать 
\end{frame}

\subsection{Scaffolding}

\begin{frame}[t]{Existing RNA-seq scaffolders}
	\begin{itemize}
		\item {\bf L\_RNA\_scaffolder} (2013) uses long transcriptome reads.
		\item {\bf AGOUTI} (2015) uses paired-end RNA-seq reads and gene predictions.
		\item {\bf rascaf} (2016) uses paired-end RNA-seq reads.
		\item {\bf P\_RNA\_scaffolder} uses paired-end RNA-seq reads, unpublished, details unknown. 
	\end{itemize}
	
	\printbibliography
			
\end{frame}

\begin{frame}[t]{L\_RNA\_scaffolder}
	\cimg{pic/LrnaScaf1}{1}	
\end{frame}

\begin{frame}[t]{L\_RNA\_scaffolder}
	\cimg{pic/LrnaScaf2}{0.9}	
\end{frame}

\begin{frame}[t]{AGOUTI}
	\cimg{pic/AGOUTI1}{0.8}	
\end{frame}

\begin{frame}[t]{AGOUTI}
	\cimg{pic/AGOUTI2}{0.75}	
\end{frame}

\begin{frame}[t]{Rascaf}
	\cimg{pic/rascaf1}{0.9}	
\end{frame}

\begin{frame}[t]{Rascaf}
	\cimg{pic/rascaf2}{1}	
\end{frame}


\begin{frame}[t]{Building a scaffold graph}
	\begin{itemize}
		\item Align RNA-seq paired-end reads
		\item Build a scaffold graph using these alignments
		\item Split every read into two parts in the middle 
		\item Align these parts independently
		\item Build graph using these parts of the reads
		\item Save graph
	\end{itemize}	
	\cimg{pic/rnaReadsCon.png}{1}
\end{frame}

%
%\begin{frame}[t]{Coordinates}
%	\cimg{pic/crd.png}{0.9}
%\end{frame}

\begin{frame}[t]{Graph simplification}
	Delete low-weight edges 
	\cimg{pic/psh.png}{1.05}
\end{frame}

\begin{frame}[t]{Graph simplification}
	Delete low-weight edges 
	\cimg{pic/pshDel.png}{1.05}
\end{frame}

\begin{frame}[t]{Graph simplification}
	Edge projection
	\cimg{pic/prj.png}{1}
\end{frame}

\begin{frame}[t]{Graph simplification}
	Edge projection
	\cimg{pic/prjDel.png}{1}
\end{frame}

\begin{frame}[t]{Graph simplification}
	Delete cycles
	\cimg{pic/cycl.png}{0.325}
\end{frame}

\begin{frame}[t]{Graph simplification}
	Delete cycles
	\cimg{pic/cyclDel.png}{0.325}
\end{frame}

\begin{frame}[t]{Graph simplification}
	Fork with big difference in weight
	\cimg{pic/bigDif.png}{0.77}
\end{frame}

\begin{frame}[t]{Graph simplification}
	Fork with big difference in weight
	\cimg{pic/bigDifDel.png}{0.77}
\end{frame}

\begin{frame}[t]{Graph simplification}
	Check coordinates order
	\cimg{pic/scafCoord.png}{1}
\end{frame}

\begin{frame}[t]{Graph simplification}
	Check coordinates order
	\cimg{pic/coordRule.png}{1}
\end{frame}

\begin{frame}[t]{Graph simplification}
	Check coordinates order
	\cimg{pic/coordRulePic.png}{1}
\end{frame}

\begin{frame}[t]{Graph simplification}
	Check coordinates order
	\cimg{pic/coordRuleDel.png}{1}
\end{frame}

\begin{frame}[t]{Graph simplification}
	Connect simple paths into scaffolds
	\cimg{pic/gline.png}{1}
\end{frame}

\begin{frame}[t]{Plans}
	\begin{itemize}
		\item use covering info
		\item find coords blocks
		\item extra validate split-50 edges
		\item ...
	\end{itemize}
\end{frame}


\begin{frame}[t]{Сomparison} 
	C.elegans
	\begin{center}
		{\small
			
		\begin{tabular}{|l|l|c|c|c|}
		\hline
		&&SRR1560107&SRR4181037&SRR5067781\\
		\hline
		\cellcolor{LightCyan}&\cellcolor{LightCyan}NGA50&\cellcolor{LightCyan}31168&\cellcolor{LightCyan}30974&\cellcolor{LightCyan}27650\\
		\hhline{~----}
		\multirow{-2}{*}{\cellcolor{LightCyan}bio\_scaf}&\cellcolor{LightCyan}mis&\cellcolor{LightCyan}550&\cellcolor{LightCyan}554&\cellcolor{LightCyan}509\\
		\hline
		\hline
		\multirow{2}{*}{P\_RNA\_scaffolder}&NGA50&28828&33645&29786\\
		\hhline{~----}
		&mis&621&750&571\\
		\hline
		\hline
		\multirow{2}{*}{rascaf}&NGA50&27116&30403&27266\\
		\hhline{~----}
		&mis&521&561&507\\
		\hline
		\hline
		\multirow{2}{*}{AGOUTI}&NGA50&25384&27047&25275\\
		\hhline{~----}
		&mis&545&613&510\\
		\hline
		\hline
		\multirow{2}{*}{contigs}&NGA50&21938&21938&21938\\
		\hhline{~----}
		&mis&352&352&352\\
		\hline
		\end{tabular}
	}
	\end{center}
\end{frame}

\section{Architecture}
\begin{frame}[t]{Architecture}
\cimg{pic/src.jpg}{0.8}
\end{frame}

\section{Results}
\begin{frame}[t]{Results and plans}
	\begin{block}{Results}
		\begin{itemize}
			\item Designed an algorithm for RNA-seq scaffolding 
			\item Developed a tool for visualization a scaffold graph
		\end{itemize}
	\end{block}
	\begin{block}{Plans}
		\begin{itemize}
			\item Compare and test against other tools on more datasets
			\item Improve performance
			\item Implement new ideas for scaffolding  
			\item Write manual and interface
			\item Write a paper
			\end{itemize}
	\end{block}
\end{frame}


\section{Thank you!}
\begin{frame}{Thank you!}
    \begin{center}
        https://github.com/olga24912/bio\_scaffolder
    \end{center}
\end{frame}
\end{document}